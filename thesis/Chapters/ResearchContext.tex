\chapter{Research Context}
\lhead{\emph{Research Context}}

Software systems are increasingly becoming regulated. While the engineers who design, build, and deploy these systems should understand the laws and regulations to which they must comply\cite{RefMassey1}, this is often not the case. Studies have demonstrated that software engineers cannot be reliably expected to determine whether a given software implementation meets a set of regulations, that is, whether the software is Legally Implementation Ready (LIR)\cite{RefMassey1}.

Legal regulations aside, most software engineers today know that software applications have security vulnerabilities that they need to eradicate. Unfortunately, they often lack the knowledge and training to effectively build security into their software\cite{RefPayne1}. One methodology that has been successfully applied to the development of secure software is \textit{formal methods} \cite{RefVoas1}. 

Considerable existing research has explored the domain of formal methods and modelling with respect to HIPAA regulations. Alshurgan et al\cite{RefAlshugran1}\cite{RefAlshugran2} have endeavoured to formalise HIPAA regulations with the goal of generating a decision engine capable of checking the validity of operations performed on sensitive PHI data, expressing the generated rules as eXtensible Access Control Markup Language (XACML). Barth et al\cite{RefBarth1} have devised a temporal-logic based approach to modelling various sets of regulations including HIPAA. Similarly, Kharbili et al\cite{RefKharbili1} have designed a DSL name CoReL, which enables automated compliance checking against sets of regulations. Maxwell et. al have devised an approach to regulation compliance checking based on artificial intelligence - HIPAA regulations are used to demonstrate their work, which allows requirements engineers to interrogate a formal HIPAA rule model. This in turn returns specific information regarding rule adherence\cite{RefMaxwell1}.

Despite successful application in several domains\cite{Woodcock1}, the software engineering community as a whole appears to remain unconvinced as to the usefulness of formal methods\cite{RefBowen1}\cite{RefSchalken1}. While results do suggest that use of formal methods is well suited to security concerns\cite{RefSchaffer1}, the approach yet to enjoy widespread industry uptake\cite{Woodcock1}.

UMLSec, a UML extension which allows the application developer to embed security related information into system design\cite{RefBest1}, has potential to narrow the gap between commonplace software development practices and the formal methods approach. Initially developed in 2002, UMLSec is a lightweight UML extension which allows specifications of security constraints on a model via UML properties and constraints. In total twenty one UMLSec stereotypes are defined; these are applied to various models and diagrams, which can subsequently be checked by a security model validator. Where applicable, application code can be generated from same models. UMLSec has been successfully used to validate protocols such as Common Electronic Purse Specifications (CEPS)\cite{RefJurgens1}.

While UMLSec's broad specification allows for general purpose security system modelling, Haqiq et. al have developed the HIPAA Modeling Method (HMM), which is specific to HIPAA regulations. This UML-based approach aims at transforming HIPAA security requirements into models in order to define the process of security policy implementation, and therefore help development teams to take into account security considerations when setting-up e-Health systems\cite{RefHaqiq1}.

Similarly, Goldstein et al. have demonstrated an holistic approach to generation of security related code using models which are augmented to express domain-specific concepts\cite{RefGoldstein1}. The approach explored - Multi-perspective Enterprise Modeling (MEMO) - relies on specialised tool support and domain specific modeling languages (DSML), enabling development of risk analysis models and metamodels. 


While model-based formal methods have been explored in the domain of HIPAA security model generation and verification, I propose that abstract models may be alternatively leveraged by the Model Driven Development (MDD) paradigm, an approach involving the direct, automated generation of production software application code from models. Rather than focusing on model-based validation of a given design, this approach is concerned with the automated generation of concrete software implementations. Unlike Goldstein's work\cite{RefGoldstein1}, which has a generic security focus, this research will explore the extension of existing and well-understood modelling languages such as UML with domain-specific constructs, for example, the power to express PHI-related constraints at the modelling level. The research will explore the opportunities generation of software artefacts such as facade patterns and persistence layer implementations from such models. 

It is proposed that automated generation of HIPAA regulated application code may yield savings in development time and costs, ultimately producing more secure code than hand-crafted alternatives. Furthermore while the possible motivations for employing model driven or formal approaches may share much in common - not least, increased confidence in a system implementation - with MDD, those errors which may inevitably creep into the implementation of a formally validated system model may potentially be avoided. Assuming the code generation tooling has been proven, it can be used to generate any number of reliable implementations, in a fraction of the time that a development team would require.


